%=======================02-713 LaTeX template, following the 15-210 template==================
%
% You don't need to use LaTeX or this template, but you must turn your homework in as
% a typeset PDF somehow.
%
% How to use:
%    1. Update your information in section "A" below
%    2. Write your answers in section "B" below. Precede answers for all 
%       parts of a question with the command "\question{n}{desc}" where n is
%       the question number and "desc" is a short, one-line description of 
%       the problem. There is no need to restate the problem.
%    3. If a question has multiple parts, precede the answer to part x with the
%       command "\part{x}".
%    4. If a problem asks you to design an algorithm, use the commands
%       \algorithm, \correctness, \runtime to precede your discussion of the 
%       description of the algorithm, its correctness, and its running time, respectively.
%    5. You can include graphics by using the command \includegraphics{FILENAME}
%
\documentclass[11pt]{article}
\usepackage{amsmath,amssymb,amsthm}
\usepackage{graphicx}
\usepackage[margin=1in]{geometry}
\usepackage{fancyhdr}
\usepackage{listings}
\lstset{language=R,
}
\setlength{\parindent}{0pt}
\setlength{\parskip}{5pt plus 1pt}
\setlength{\headheight}{13.6pt}
\newcommand\question[2]{\vspace{.25in}\hrule\textbf{#1: #2}}
\renewcommand\part[1]{\vspace{.10in}\textbf{(#1)}}
\newcommand\algorithm{\vspace{.10in}\textbf{Algorithm: }}
\newcommand\correctness{\vspace{.10in}\textbf{Correctness: }}
\newcommand\runtime{\vspace{.10in}\textbf{Running time: }}
\pagestyle{fancyplain}
\lhead{\textbf{\NAME}}
\chead{\textbf{HW \HWNUM}}
\rhead{\today}
\begin{document}\raggedright
%Section A==============Change the values below to match your information==================
\newcommand\NAME{Jonathan Lopez}  % your name
\newcommand\HWNUM{1}              % the homework number
%Section B==============Put your answers to the questions below here=======================

% no need to restate the problem --- the graders know which problem is which,
% but replacing "The First Problem" with a short phrase will help you remember
% which problem this is when you read over your homeworks to study.

\question{2.2.4}{Suppose that two cards are dealt from a standard 52-card poker
deck. Let $A$ be the event that the sum of the two cards is 8 (assume that aces
have a numerical value of 1). How many outcomes are in $A$?} 
\\ A + 7 = 8, 16 outcomes
\\ 2 + 6 = 8, 16 outcomes
\\ 3 + 5 = 8, 16 outcomes
\\ 4 + 4 = 8, 6 outcomes
\\ 54 total outcomes in $A$


\question{2.2.11}{A woman has her purse snatched by two teenagers. She is 
subsequently shown a police lineup consisting of five suspects, including
the two perpetrators. What is the sample space associated with the experiment
"Woman picks two suspects out of lineup"? Which outcomes are in the event $A$ She makes at least one incorrect identification?}
\\ $S$=\{Woman makes two correct identifications,
\\ Woman makes one correct identifications,
\\ Woman makes no correct identifications\}
\\ $A$ = \{Woman makes one correct identification, Woman makes no correct identifications\}



\question{2.2.28}{Let events $A$ and $B$ and sample space $S$ 
    be defined as the following intervals: \\
$S = \{x: 0 \leq\ x\leq\ 10\}$ \\
$A = \{x: 0 <\ x <\ 5\}$ \\
$B = \{x: 3 \leq\ x\leq\ 7\}$ \\
Characterize the following events:}
\\ \part{a} {$A^C = \{0,5,6,7,8,9,10\}$}
\\ \part{b} {$A \cap\ B = \{3,4\}$}
\\ \part{c} {$A \cup\ B = \{1,2,3,4,5,6,7\}$}
\\ \part{d} {$A \cap\ B^C = \{1,2\}$}
\\ \part{e} {$A^C \cup\ B = \{0,3,4,5,6,7,8,9,10\}$}
\\ \part{f} {$A^C \cap\ B^C = \{0,8,9,10\}$}


\question{2.2.40}{For two events $A$ and $B$ defined on a sample
    space $S$,  
    $N (A \cap\ B^{C} )=15$, 
    $N (A^C \cap\ B)=50$, and
    $N (A \cap\ B)=2$. 
    Given that 
    $N(S)=120$
, how many outcomes belong to neither $A$ nor $B$?}

\\ $N(A \cup B) = N(A \cap B^C) + N(A^C \cap B) + N(A \cap B) = 15 + 50 + 2 = 67$
\\ $N((A \cup B)^{C}) = N(S) - N(A \cup B) = 120 - 67 = 53$


    \question{2.3.2}{Let $A$ and $B$ be any two events defined on
    $S$. Suppose that $P(A)=0.4$, $P(B)=0.5$, and $P(A \cap B)=0.1$.
What is the probability that $A$ or $B$ but not both occur?}
\\ $P(A) + P(B) - P(A \cap B) = 0.4 + 0.5 - 0.1 = 0.8$


\question{2.3.12}{Events $A_1$ and $A_2$ are such that $A_1 \cup A_2 = 2$
and $A_1 \cap A_2 = \emptyset$. Find $p_2$ if $P(A_1)=p_1$, $P(A_2)=p_2$,
and $3p_1 - p_2 = \frac{1}{2}$.}
\\ $p_1 + p_2 = 2$
\\ $3p_1 - p_2 = 0.5$
\\ $4p_1 + 0 = 2.5$
\\ $p_1 = \frac{2.5}{4} = 0.625$
\\ $p_2 = 2 - 0.625 = 1.375$


\question{2.3.16}{Two dice are tossed. Assume that each possible outcome has
a $\frac{1}{36}$ probability. Let $A$ be the event that the sum of the faces
showing is 6, and let $B$ be the event that the face showing on one die is
twice the face showing on the other. Calculate $P(A \cap B^C)$.}
\\ $A = \{(1,6), (2,4), (3,3), (4,2), (6,1)\}$
\\ $B = \{(1,2), (2,4), (3,6)\}$
\\ $A \cap B^C = \{(1,6), (3,3), (6,1)\}$
\\ $P(A \cap B^C) = \frac{N(A \cap B^C)}{36} = \frac{3}{36} = \frac{1}{12}$



\question{2.4.7}{An urn contains one red chip and one white chip.
One chip is drawn at random. If the chip selected is red, that chip together
with two additional red chips are put back into the urn. If a white chip
is drawn, the chip is returned to the urn. Then a second chip is drawn.
What is the probability that both selections are red?}
\\ $(\frac{1}{2})  (\frac{3}{4}) = \frac{3}{8}$


\question{2.4.10}{Suppose events $A$ and $B$ are such that $P(A \cap B) = 0.1$
and $P((A \cup B)^C ) = 0.3$. If $P(A)=0.2$, what does 
$P[(A \cap B)|(A \cup B)^C ]$ equal?}
\\ $P[(A \cap B)|(A \cup B)^C ] = \frac{P((A \cap B) \cap (A \cup B)^C)}{P((A \cup B)^C)}$
\\ $P((A \cup B) \cap (A \cup B)^C) = 0$
\\ $P[(A \cap B)|(A \cup B)^C ] = \frac{0}{P((A \cup B)^C)} = 0$


\question{2.4.16}{Given that $P(A)+P(B)=0.9$, $P(A|B)=0.5$,
and $P(B|A)=0.4$, find $P(A)$.}
\\ $P(A|B) = \frac{P(A \cap B)}{P(B)}=0.5 => (0.5)P(B) = P(A \cap B)$
\\ $P(B|A) = \frac{P(A \cap B)}{P(A)}=0.4 => (0.4)P(A) = P(A \cap B)$
\\ $(0.5)P(B) = (0.4)P(A) => P(B) = \frac{0.4}{0.5}P(A) = (0.8)P(A)$
\\ $P(A) + (0.8)P(A) = (1.8)P(A) = 0.9 => P(A) = \frac{0.9}{1.8} = 0.5$


% \question{2.4.23}{Your favorite college football team has had a good
% season so far but they need to win at least two of their last four
% games to qualify fo a New Year's Day bowl bid.
% Oddsmakers estimate the team's probabilities of winning each of the last four
% games to be 0.60, 0.50, 0.40, and 0.70, respectively.}
% 
% \part{a} {What are the chances that you will get to watch your team play on Jan. 1?}
% \\ $1 - (1-0.60)(1-0.50)(1-0.40)(1-0.70) = 1 - (0.40)(0.50)(0.60)(0.70) = 1 - 
% 
% \part{b} {Is the probability that your team wins all four games given that they
% have won at least three games equal to the probability that they win the 
% fourth game? Explain.}
% 
% \part{c} {Is the probability that your team wins all four games given that they
% won the first three games equal to the probability that they win the fourth game?}
% 
% 
% \question{2.4.46}{Brett and Maro have each thought about murdering their rich 
% Uncle Basil in hopes of claiming their inheritance a bit early. Hoping to take
% advantage of Basil's pedilection for immoderate desserts, Brett has put rat
% poison in the cherries flambe; Margo, unaware of Brett's activities, has laced
% the chocolate mousse with cyanide. Given the amounts likely to be eaten, the
% probability of the rat poison being fatal is 0.60; the cyanide, 0.90. Based
% on other dinners where Basil was presented with the same dessert options, we can
% assume that he has a 50\% chance of asking for the cherries flambe, a 40\% chance
% of ordering the chocolate mousse, and a 10\% chance of skipping dessert altogether.
% No sooner are the dishes cleared away than Basil drops dead. In the absence of any
% other evidence, who should be conidered the prime suspect?}
% 
% 
% \question{2.5.5}{Dana and Cathy are playing tennis. The probability that Dana wins
% at least one out of two games is 0.3. What is the probability that Dana wins at least
% one out of four?}
% \\ $P(\frac{0}{2}) + P(\frac{1}{2}) + P(\frac{2}{2}) = 1$
% \\ $P(\frac{1}{2}) + P(\frac{2}{2}) = 0.3$
% 
% 
% \question{2.5.16}{On her way to work, a commuter encounters four traffic signals.
% Assume that the distance between each of the four is sufficiently great that the
% probability of getting a green light at any intersection is independent of what happened at any previous intersection. The first two lights are green for forty seconds of each
% minute; the last two, for thirty seconds of each minute. What is the probability that
% the commuter has to stop at least three times?}
% 
% 
% \question{2.5.32}{What is the smallest number of switches wired in parallel that
% will give a probability of at least 0.98 that a circuit will be completed? Assume
% that each switch operates independently and will function properly 60\% of the time.}


\question{Monte Carlo Exercise}{John has integers 1:10. He randomly draws 5 without
replacement and reasons that he could estimate the 80th percentile of his 10 integers,
the value 8, by taking the 2nd largest sampled value; that is the 4th value in order
from smallest to largest.}

\part{a} {Applying this approach repetitively, what proportion of the time will
he accurately estimate the value 8?}
\begin{lstlisting}
estimate80th <- function(numberOfTrials) {
    counter <- 0
    for (trial in 1:numberOfTrials) {
        # This will create a sorted list of numbers from 1-10
        # and sample 5 at random without replacement
        result <- sort(sample(1:10, 5, replace=FALSE))
        if (result[4] == 8) counter <- counter + 1
    }
    return(counter/numberOfTrials)
}

# Let's run 100,000 trials
print(estimate80th(100000)) 
# Roughly 27% correctly estimates

\end{lstlisting}

\part{b} {Underestimate?}
\begin{lstlisting}
underestimate80th <- function(numberOfTrials) {
    counter <- 0
    for (trial in 1:numberOfTrials) {
        # This will create a sorted list of numbers from 1-10
        # and sample 5 at random without replacement
        result <- sort(sample(1:10, 5, replace=FALSE))
        if (result[4] <  8) counter <- counter + 1
    }
    return(counter/numberOfTrials)
}
# Let's run 100,000 trials
print(underestimate80th(100000))
# Roughly 50% of time will underestimate
\end{lstlisting}

\part{c} {Overestimate?}
\begin{lstlisting}
overestimate80th <- function(numberOfTrials) {
    counter <- 0
    for (trial in 1:numberOfTrials) {
        # This will create a sorted list of numbers from 1-10
        # and sample 5 at random without replacement
        result <- sort(sample(1:10, 5, replace=FALSE))
        if (result[4] > 8) counter <- counter + 1
    }
    return(counter/numberOfTrials)
}
# Let's run 100,000 trials
print(overestimate80th(100000))
# Roughly 22% of time will overestimate
\end{lstlisting}
\end{document}
