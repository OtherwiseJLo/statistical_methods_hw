\documentclass[11pt]{article}
\usepackage{amsmath,amssymb,amsthm}
\usepackage{graphicx}
\usepackage[margin=1in]{geometry}
\usepackage{fancyhdr}
\setlength{\parindent}{0pt}
\setlength{\parskip}{5pt plus 1pt}
\setlength{\headheight}{13.6pt}
\newcommand\question[2]{\vspace{.25in}\hrule\textbf{#1: #2}\vspace{.5em}\vspace{.10in}}
\renewcommand\part[1]{\vspace{.10in}\textbf{(#1)}}
\pagestyle{fancyplain}
\lhead{\textbf{\NAME }}
\chead{\textbf{HW\HWNUM}}
\rhead{\today}
\begin{document}\raggedright
%Section A==============Change the values below to match your information==================
\newcommand\NAME{Jonathan Lopez}  % your name
\newcommand\HWNUM{6}              % the homework number
%Section B==============Put your answers to the questions below here=======================



\question{4.2.4}
{A chromosome mutation linked with colorblindness is known to occur, on the average,
once in every ten thousand births}
\\
\part{a}{Approximate the probability that exactly three of the next twenty thousand
babies born will have the mutation.}
\\
\part{b}{How many babies out of the next twenty thousand would have to be born with
the mutation to convince you that "one in ten thousand" estimate is too low?}
\\


\question{4.2.12}
{Midwestern Skies books ten commuter flights each week. Passenger totals are much
the same from week to week, as are the number of pieces of luggage that are checked.
Listed in the following table are the numbers of bags that were lost during each of
the first forty weeks in 2009. Do the figures support the presumption that the number
of bags lost by Midwestern during a typical week is a Poisson random variable?}
\\


\question{4.2.28}
{Fifty spotlights have just been installed in an outdoor security system. According 
to the manufacturer's specification, these particular lights are expected to burn
out at a rate of 1.1 per one hundered hours. What is the expected number of bulbs
that will fail to last for at least seventy-five hours?}
\\


\question{4.3.2}
{Let Z be a standard normal random variable. Use Appendix Table A.1 to find the
numerical value for each of the following probabilities. Show each of your answers
as an area under $f_{Z}(z)$.}
\\
\part{a}{$P(0 \leq Z \leq 2.07$}
\\
\part{b}{$P(-0.64 \leq Z \leq -0.11$}
\\
\part{c}{$P(Z > -1.06$}
\\
\part{d}{$P(Z < -2.33$}
\\
\part{e}{$P(Z \geq 4.61$}
\\


\question{4.3.14}
{A sell-out crowd of 42,000 is epxected at Cleaveland's Jacob's Field for next
Tuesday's game against the Baltimore's Orioles, the last before a long road trip.
The ballpark's concession}
\\


\question{4.3.18}
{Suppose $X_{1}$,$X_{2}$,$X_{3}$, and $X_{4}$ are independent Poisson random
variables, each with parameter $\lambda = 3$. Let $S=X_{1}+X_{2}+X_{3}+X_{4}$}
\\
\part{a}{Use the Central Limit Theorem to approximate the probability that
$13 \leq S \leq 14$.}
\\
\part{b}{Calculate the exact probability that $13 \leq S \leq 14$.}
\\


\question{4.3.22}
{A large computer chip manufacturing plant under construction in Westbank is expected
to result in an additional fourteen hundren children in the country's public school
system once the permanent workforce arrives. Any child with an IQ under 80 or over
135 will require individualized instruction that will cost the city an additional 
\$1750 per year. How much money should Westbank anticipate spending next year to meet
the needs of its new special ed students? Assume that IQ cores are noramlly distrubuted
with a mean($\mu$) of 100 and a standard deviation($\sigma$) of 16.}
\\


\question{4.4.6}
{Suppose that the cdf for a geometric random variable is given by 
$F_{X}(t) = P(X \leq t) = 1 - (1-p)^{[t]}$, where $[t]$ denotes the greatest integer
in $t,t \geq 0$.}
\\


\question{4.4.10}
{Suppse that the random variables $X_{1}$ and $X_{2}$ have mfgs 
$M_{X_{1}}(t) = \frac{ \frac{1}{2}e^{t} }{1-(1-\frac{1}{2})e^{t}}$ 
and
$M_{X_{2}}(t) = \frac{ \frac{1}{4}e^{t} }{1-(1-\frac{1}{4})^{t}}$
, respectively. Let $X=X_{1}+X_{2}$. Does $X$ have a geometric distribution?
Assume that $X_{1}$ and $X_{2}$ are independent.}




\question{4.5.5}
{For some negative binomial variable whose pdf is given by Equation 4.5.1, find $E(X)$
directly by evaluating
$\sum_{k=r}^{\infty}k \binom{k-1}{r-1} p^{r}(1-p)^{k-r}$.
\emph{Hint: Reduce the sum to one involving negative binomial properties wih
parameters $r+1$ and $p$.}}


\question{4.6.2}
{A service contact on new university computer system provides stenty-four free repair
 calls from a technician. Suppose the technician is required, on average, three times
 a month. What is the average time it will take for the service contract to be 
 fulfulled?}


\question{4.6.6}
{Prove that $\Gamma (\frac{1}{2}) = \sqrt{\pi}$.
\emph{(Hint: consider $E(Z^{2})$, where $Z$ is a standard normal random variable.)}}


\question{4.6.9}
{Differentiate the gamma moment-generating function to verify the forumulas for
$E(Y)$ and $Var(Y)$ given in Theorem 4.6.3.}


\end{document}
